%% start of file `template.tex'.
%% Copyright 2006-2013 Xavier Danaux (xdanaux@gmail.com).
%
% This work may be distributed and/or modified under the
% conditions of the LaTeX Project Public License version 1.3c,
% available at http://www.latex-project.org/lppl/.


\documentclass[11pt,letterpaper,sans]{moderncv}        % possible options include font size ('10pt', '11pt' and '12pt'), paper size ('a4paper', 'letterpaper', 'a5paper', 'legalpaper', 'executivepaper' and 'landscape') and font family ('sans' and 'roman')

% moderncv themes
\moderncvstyle{casual}                             % style options are 'casual' (default), 'classic', 'oldstyle' and 'banking'
\moderncvcolor{blue}                               % color options 'blue' (default), 'orange', 'green', 'red', 'purple', 'grey' and 'black'
%\renewcommand{\familydefault}{\sfdefault}         % to set the default font; use '\sfdefault' for the default sans serif font, '\rmdefault' for the default roman one, or any tex font name
%\nopagenumbers{}                                  % uncomment to suppress automatic page numbering for CVs longer than one page

%\usepackage{fontspec}
%\setmainfont{Fontin} % Main document font
%\usepackage{fontawesome}
\usepackage{fontspec} % For loading fonts
%\defaultfontfeatures{Mapping=tex-text}
%\setmainfont[SmallCapsFont = Fontin SmallCaps]{Fontin} % Main document font
\setmainfont{Calibri} % Main document font

% character encoding
\usepackage[utf8]{inputenc}                       % if you are not using xelatex ou lualatex, replace by the encoding you are using
%\usepackage{CJKutf8}                              % if you need to use CJK to typeset your resume in Chinese, Japanese or Korean

% adjust the page margins
\usepackage[scale=0.75]{geometry}
%\setlength{\hintscolumnwidth}{3cm}                % if you want to change the width of the column with the dates
%\setlength{\makecvtitlenamewidth}{10cm}           % for the 'classic' style, if you want to force the width allocated to your name and avoid line breaks. be careful though, the length is normally calculated to avoid any overlap with your personal info; use this at your own typographical risks...
\usepackage{siunitx}
\DeclareSIUnit\year{yr}
\usepackage[version=4]{mhchem}
\usepackage{tabularx}
\setlength{\tabcolsep}{12pt}

% personal data
\name{Michinari}{Sakai}
\title{Resumé title}                               % optional, remove / comment the line if not wanted
\address{1235 Solano Ave Apt 10}{Albany, CA 94706}{USA}% optional, remove / comment the line if not wanted; the "postcode city" and and "country" arguments can be omitted or provided empty
\phone[mobile]{+1~(808)~206~4357}                   % optional, remove / comment the line if not wanted
%\phone[fixed]{+2~(345)~678~901}                    % optional, remove / comment the line if not wanted
%\phone[fax]{+3~(456)~789~012}                      % optional, remove / comment the line if not wanted
\email{michsakai@berkeley.edu}                               % optional, remove / comment the line if not wanted
%\homepage{www.johndoe.com}                         % optional, remove / comment the line if not wanted
%\extrainfo{additional information}                 % optional, remove / comment the line if not wanted
%\photo[64pt][0.4pt]{picture}                       % optional, remove / comment the line if not wanted; '64pt' is the height the picture must be resized to, 0.4pt is the thickness of the frame around it (put it to 0pt for no frame) and 'picture' is the name of the picture file
%\quote{Some quote}                                 % optional, remove / comment the line if not wanted

% to show numerical labels in the bibliography (default is to show no labels); only useful if you make citations in your resume
%\makeatletter
%\renewcommand*{\bibliographyitemlabel}{\@biblabel{\arabic{enumiv}}}
%\makeatother
%\renewcommand*{\bibliographyitemlabel}{[\arabic{enumiv}]}% CONSIDER REPLACING THE ABOVE BY THIS

% bibliography with mutiple entries
%\usepackage{multibib}
%\newcites{book,misc}{{Books},{Others}}
%----------------------------------------------------------------------------------
%            content
%----------------------------------------------------------------------------------
\begin{document}
%-----       letter       ---------------------------------------------------------
% recipient data
%\recipient{SNOLAB}{Creighton Mine \#9\\1039 Regional Road 24\\Lively, ON P3Y 1N2\\Canada}
\recipient{Advanced Concepts and Technologies Group}
{Combat Systems Branch of the Air and Missile Defense Sector}
\date{July 27, 2019}
\opening{Dear Team Hiring Manager,}
%\closing{Best wishes,}
\closing{Thank you for your time,}
%\enclosure[Attached]{curriculum vit\ae{}, research statement, publication list, list of referees}          % use an optional argument to use a string other than "Enclosure", or redefine \enclname
\makelettertitle

As a budding and motivated physicist, I am extremely interested in the Research
Associate position to join your team in the study of nuclear weapons effects.
I would like to apply for the position.

I am currently a post-doctoral scholar at the University of California, Berkeley
with expertise in particle transport simulations working with the CUORE
(Cryogenic Underground Observatory for Rare Events) experiment.

%CUORE is an unprecedented tonne-scale
%bolometric search for lepton number violation in \ce{^{130}Te}. The experiment
%was successfully upgraded last year increasing its detector mass by a factor of
%almost 20. A paper of our first $0\nu\beta\beta$ analysis with this increased
%mass was submitted for publication to PRL in late 2017.
%(https://arxiv.org/abs/1710.07988).

%The sensitivity of CUORE is expected to start probing the inverted neutrino
%mass hierarchy phase space of the effective Majorana neutrino mass within the
%next 5 years of data taking. However, In order to advance further, we are
%confronted with the challenge of further reducing our background in the region
%of interest by more than \num{2} orders of magnitude that the current level of \SI{\sim 0.01}{\text{count}\per\kilo\electronvolt\per\kilo\gram\per\year}. I
%am currently spearheading the effort on precision modeling of the alpha
%background spectrum to better understand our radioactive backgrounds to meet
%this challenge.

%I received my Ph.D. in experimental neutrino physics from the University of
%Hawaii at Manoa in May of 2016. During my graduate studies, I worked on event
%track reconstruction and particle ID techniques in KamLAND (Kamioka Liquid
%Scintillator Antineutrino Detector), a monolithic liquid scintillator neutrino
%detector in Kamioka, Japan. Through my work, I single handedly opened the doors
%to a never before explored capability of scintillator detectors to conduct
%indirect dark matter searches by looking for directional neutrino signals from
%the core of the Sun and Earth. According to my understanding, my work is the
%first ever physics application of directional neutrino reconstruction in
%scintillator. A paper for this is currently under preparation.
%
%Nevertheless, topological event reconstruction in scintillator is fundamentally
%limited because the reconstruction is done using the smeared information of
%isotropically emitted scintillation photons collected by photomultiplier tubes
%(PMTs) along the perimeter of the detector volume. Liquid Argon
%Time-Projection-Chamber (LArTPC) technology that will be used by the DUNE (Deep
%Underground Neutrino Experiment) detector at Sanford Lab is the next
%generation endeavor to employ topological event imaging at an unprecedented
%level of precision, using the ability to drift ionization charge to a read-out
%plane. Although there has been much effort to demonstrate the feasibility of
%the detector technology itself, development of reliable event imaging
%algorithms will be crucial to the success of DUNE. I believe that my experience
%with topological event imaging in an extremely difficult medium, such as
%scintillator, puts me in a unique position to play a leading role in the effort
%to develop reliable reconstruction algorithms in LArTPCs.

%In addition, throughout my academic career, I have been heavily involved in
%teaching and mentoring students. My philosophy for teaching is motivation
%through curiosity and mastering through repetition. When teaching at the
%undergraduate level, it is especially important to never get bogged down by the
%equation or numbers. When I begin a class or introduce a new concept to
%students, I never begin with equations. I always first show them the experiment
%itself or introduce the context in an illustrative way. This is to engage the
%students' interest and entice their curiosity to develop a long lasting
%motivation before starting to learn. Repetition is the key to mastering a
%topic. No matter how gifted one may be, no one able to master everything
%instantaneously in a single trial. On the other hand, this also means that even
%if one is not naturally gifted at something, it can be mastered through
%diligent repetition.

%In society today, there is an unfortunate misconception that Physics is a
%difficult subject and that it is only for \textit{smart} students. Nothing
%could be further from the truth. When students feel that they are not good
%enough or smart enough to follow the class, it is important to let them know
%that Physics is not about intelligence, but about curiosity driven diligence
%and hard work. All professional Physicists today stand on the shoulders of the
%giants of the past, and almost no research is done solely alone, but through
%collaboration and teamwork. No one is \textit{smart} enough to do all the work
%by one's self. I usually use myself as an example. I could not attend a leading
%world-class research institute during my undergraduate school days, but I
%believe through diligence and hard work, I have had the opportunity to work
%with some of the most talented and innovative Physicists in the field.

%In conclusion, my innovative endeavors in topological neutrino event
%reconstruction, as well as my experience in successfully teaching/mentoring
%students at both the undergraduate and graduate levels, makes me a unique
%candidate to apply for your position. I believe I can make a significant impact
%to your team of academic and scientific prowess at SD School of Mines \&
%Technology.

With regard to my ability to meet the specific requirements of this job:
\noindent\begin{tabularx}{\linewidth}{@{{}\textbullet\enskip}X@{\quad}@{}}
	\textbf{Education and experience:} Received PhD in 2016 with 6 years of
	experience working on particle transport simulations and algorithm
	development for novel dark matter detection technologies.\\
	\textbf{Ability to obtain secret clearance:} US citizen by birth.\\
	\textbf{Familiar programming languages:} C, C++, Python, Mathematica, Fortran, SolidWorks.\\
	\textbf{Knowledge of nuclear physics, particle physics:}
	8 years of experience in particle physics and 2 years of experience in
	nuclear physics both working with particle/radiation interactions.\\
	\textbf{Experience in radiation transport code:} 8 years of experience
	working with Geant4.\\
	\textbf{Experience with transport problems in atmosphere or within
	structures:} Simulation work with CUORE involves nuclear decay radiation
	through various materials such as steel, lead, copper, Teflon. Other past
	work involved atmospheric muon and their decay daughter simulations with the
	KamLAND neutrino detector experiment in Japan. Was responsible for
	debugging volume overlaps in simulations with 1000's of parts to mitigate
	``ghost'' effects of photon leakage in detector. Work at UCLA involved
	study of particle transport step size on energy deposition in materials.\\
	\textbf{Familiarity with command line programming in Linux environment:}
	Was Linux server administrator for research group during 3 years of
	graduate work. All of my professional career has involved working with the
	command line in a Linux environment.\\
	\textbf{Demonstrated written and verbal communication skills:}
	Contributed work published in journals such as International Journal of
	Modern Physics A, Physical Review Letters, European Physical Journal. Talks
	given at various institutions such as Lawrence Livermore National
	Laboratory, Fermilab, Argonne National Laboratory.\\
	%\textbf{Good interpersonal skills and teamwork:} Worked in 3 multinational
	%collaborations in Japan, Italy, and US. Mentored PhD-level students and
	%increased productivity by organizing group initiatives.\\
	%\textbf{Good presentation skills:} Gave a number of talks and seminars at
	%conferences world-wide.\\
	%\textbf{Familiarity with programming languages and tools:} Proficient in
	%ROOT, C, C++, Python, Mathematica, Fortran, AutoCad, Pads. Experience with
	%Perl, R, VHDL.\\
\end{tabularx}

%In conclusion, my innovative endeavors in topological neutrino event
%reconstruction, as well as experience being involved in the commissioning phase
%of two experiments, makes me a unique candidate to apply for your position. I
%believe I can make a significant impact to your team of academic and scientific
%prowess at The University of Colorado.

I would appreciate an opportunity to meet and discuss my application at an
interview. I have also sent my resume and other relevant documents for your
consideration. Please feel free to let me know if you have any questions.

\makeletterclosing

\end{document}


%% end of file `template.tex'.
