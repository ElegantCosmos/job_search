%% start of file `template.tex'.
%% Copyright 2006-2013 Xavier Danaux (xdanaux@gmail.com).
%
% This work may be distributed and/or modified under the
% conditions of the LaTeX Project Public License version 1.3c,
% available at http://www.latex-project.org/lppl/.


\documentclass[11pt]{moderncv}        % possible options include font size ('10pt', '11pt' and '12pt'), paper size ('a4paper', 'letterpaper', 'a5paper', 'legalpaper', 'executivepaper' and 'landscape') and font family ('sans' and 'roman')


% moderncv themes
\moderncvstyle{casual}                             % style options are 'casual' (default), 'classic', 'oldstyle' and 'banking'
\moderncvcolor{blue}                               % color options 'blue' (default), 'orange', 'green', 'red', 'purple', 'grey' and 'black'
%\renewcommand{\familydefault}{\sfdefault}         % to set the default font; use '\sfdefault' for the default sans serif font, '\rmdefault' for the default roman one, or any tex font name
%\nopagenumbers{}                                  % uncomment to suppress automatic page numbering for CVs longer than one page

\usepackage{fontspec}
%\setmainfont{Fontin} % Main document font
%\usepackage{fontawesome}
%\usepackage{fontspec} % For loading fonts
%\defaultfontfeatures{Mapping=tex-text}
%\setmainfont[SmallCapsFont = Fontin SmallCaps]{Fontin} % Main document font
\setmainfont{Times New Roman} % Main document font

% character encoding
\usepackage[utf8]{inputenc}                       % if you are not using xelatex ou lualatex, replace by the encoding you are using
%\usepackage{CJKutf8}                              % if you need to use CJK to typeset your resume in Chinese, Japanese or Korean

% adjust the page margins
\usepackage[scale=0.75]{geometry}
%\setlength{\hintscolumnwidth}{3cm}                % if you want to change the width of the column with the dates
%\setlength{\makecvtitlenamewidth}{10cm}           % for the 'classic' style, if you want to force the width allocated to your name and avoid line breaks. be careful though, the length is normally calculated to avoid any overlap with your personal info; use this at your own typographical risks...
\usepackage{siunitx}
\DeclareSIUnit\year{yr}
\usepackage[version=4]{mhchem}
\usepackage{tabularx}
\setlength{\tabcolsep}{12pt}

\usepackage{ragged2e}

% personal data
\name{Michinari}{Sakai}
\title{Resumé title}                               % optional, remove / comment the line if not wanted
\address{1235 Solano Ave Apt 10}{Albany, CA 94706}{USA}% optional, remove / comment the line if not wanted; the "postcode city" and and "country" arguments can be omitted or provided empty
\phone[mobile]{808-206-4357}                   % optional, remove / comment the line if not wanted
%\phone[fixed]{+2~(345)~678~901}                    % optional, remove / comment the line if not wanted
%\phone[fax]{+3~(456)~789~012}                      % optional, remove / comment the line if not wanted
\email{michsakai@gmail.com}                               % optional, remove / comment the line if not wanted
%\homepage{www.johndoe.com}                         % optional, remove / comment the line if not wanted
%\extrainfo{additional information}                 % optional, remove / comment the line if not wanted
%\photo[64pt][0.4pt]{picture}                       % optional, remove / comment the line if not wanted; '64pt' is the height the picture must be resized to, 0.4pt is the thickness of the frame around it (put it to 0pt for no frame) and 'picture' is the name of the picture file
%\quote{Some quote}                                 % optional, remove / comment the line if not wanted

% to show numerical labels in the bibliography (default is to show no labels); only useful if you make citations in your resume
%\makeatletter
%\renewcommand*{\bibliographyitemlabel}{\@biblabel{\arabic{enumiv}}}
%\makeatother
%\renewcommand*{\bibliographyitemlabel}{[\arabic{enumiv}]}% CONSIDER REPLACING THE ABOVE BY THIS

% bibliography with mutiple entries
%\usepackage{multibib}
%\newcites{book,misc}{{Books},{Others}}
%----------------------------------------------------------------------------------
%            content
%----------------------------------------------------------------------------------
\begin{document}
%-----       letter       ---------------------------------------------------------
% recipient data
%\recipient{SNOLAB}{Creighton Mine \#9\\1039 Regional Road 24\\Lively, ON P3Y 1N2\\Canada}
\recipient{Nuclear and Particle Physics Group }{Nuclear and Chemical Sciences Division\\Lawrence Livermore National Laboratory}
\date{January 4, 2020}
\opening{Dear Team Hiring Manager,}
%\closing{Best wishes,}
\closing{Thank you for your time,}
%\enclosure[Attached]{curriculum vit\ae{}, research statement, publication list, list of referees}          % use an optional argument to use a string other than "Enclosure", or redefine \enclname
\makelettertitle

\justifying
As a motivated physicist, I am interested in joining your team to develop and
conduct low-energy nuclear-physics experiments that use the surrogate-reaction
method to measure important nuclear-reaction cross sections.
I would like to apply for the advertised post-doctoral researcher position
106535.

I am currently a post-doctoral researcher at the University of California,
Berkeley with expertise in gamma and charged particle detection through highly
segmented calorimeters and Geant4 detector physics simulations working in the
field of nuclear/particle physics.

With regard to my ability to meet the specific requirements of this job:\\
%\noindent\begin{tabularx}{\linewidth}{@{{}\textbullet\enskip}X@{\quad}@{}}
\noindent\begin{tabularx}{\linewidth}{@{{}\textbullet\enskip}X@{\quad}@{}}
	\textbf{Education:} PhD in experimental particle physics (2016) with
	emphasis on Geant4 particle transport simulations, and algorithm
	development for novel particle detection technologies.\\
	\textbf{Knowledge of accelerator-based nuclear-physics experiments, neutron
	irradiations:}
	Graduate work involves reconstruction of neutrinos interacting with target
	nuclei in KamLAND using the neutrino beam from the J-PARC facility in
	Japan. Author of Geant4 simulation code of portable scintillator detector
	mini-TimeCube that was deployed at the research reactor at NIST to study
	its feasibility as a neutron camera.\\
	\textbf{Experience with highly segmented detectors, spectroscopy, and their
	associated data analysis:} Currently data analysis lead using
	C++/ROOT/Python for investigation of alpha decay spectrum of nuclear decay
	backgrounds in the highly segmented calorimeter detector CUORE.\\
	\textbf{Knowledge of programming in C/C++/ROOT and/or other programming
	languages:} 8 years of experience with C/C++/ROOT analyzing large
	physics data sets. 4 years of Python experience.\\
	\textbf{Experience with Geant4:} Author of detector simulation code for
	mini-TimeCube scintillator detector project. Performed Geant4 simulations
	in the highly segmented calorimeter detector CUORE to reconstruct the
	energy spectrum above \SI{2.5}{MeV}. Tutored 3 PhD-level students during
	weekly Geant4 tutorials at UCLA for a semester.\\
	\textbf{Experience with digital and analog electronics for data acquisition
	systems, specifically NIM:} Current project involves using NIM modules to
	amplify and discriminate photomultiplier tube signals in conjunction with
	custom LabView DAQ program to extract, filter, and save raw data.\\
	\textbf{Documented publication record and proficiency in verbal/written
	communication skills:} 
	Contributed work published in various journals. Talks given at institutions
	such as Los Alamos National Laboratory, Sanford Underground Research
	Facility (SURF), Argonne National Laboratory, DNP (Division of Nuclear
	Physics) conferences.\\
	\textbf{Experience working collaboratively as part of a team:}
	Worked in 3 multinational collaborations in Japan, Italy, and US. Original
	work successfully interfaced with larger collaboration.\\
\end{tabularx}

I would appreciate an opportunity to meet and discuss my application at an
interview. I have also sent my resume for your consideration. Please feel free
to let me know if you have any questions.

\makeletterclosing

\end{document}


%% end of file `template.tex'.
