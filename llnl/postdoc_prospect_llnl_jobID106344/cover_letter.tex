%% start of file `template.tex'.
%% Copyright 2006-2013 Xavier Danaux (xdanaux@gmail.com).
%
% This work may be distributed and/or modified under the
% conditions of the LaTeX Project Public License version 1.3c,
% available at http://www.latex-project.org/lppl/.


\documentclass[11pt]{moderncv}        % possible options include font size ('10pt', '11pt' and '12pt'), paper size ('a4paper', 'letterpaper', 'a5paper', 'legalpaper', 'executivepaper' and 'landscape') and font family ('sans' and 'roman')


% moderncv themes
\moderncvstyle{casual}                             % style options are 'casual' (default), 'classic', 'oldstyle' and 'banking'
\moderncvcolor{blue}                               % color options 'blue' (default), 'orange', 'green', 'red', 'purple', 'grey' and 'black'
%\renewcommand{\familydefault}{\sfdefault}         % to set the default font; use '\sfdefault' for the default sans serif font, '\rmdefault' for the default roman one, or any tex font name
%\nopagenumbers{}                                  % uncomment to suppress automatic page numbering for CVs longer than one page

\usepackage{fontspec}
%\setmainfont{Fontin} % Main document font
%\usepackage{fontawesome}
%\usepackage{fontspec} % For loading fonts
%\defaultfontfeatures{Mapping=tex-text}
%\setmainfont[SmallCapsFont = Fontin SmallCaps]{Fontin} % Main document font
\setmainfont{Times New Roman} % Main document font

% character encoding
\usepackage[utf8]{inputenc}                       % if you are not using xelatex ou lualatex, replace by the encoding you are using
%\usepackage{CJKutf8}                              % if you need to use CJK to typeset your resume in Chinese, Japanese or Korean

% adjust the page margins
\usepackage[scale=0.75]{geometry}
%\setlength{\hintscolumnwidth}{3cm}                % if you want to change the width of the column with the dates
%\setlength{\makecvtitlenamewidth}{10cm}           % for the 'classic' style, if you want to force the width allocated to your name and avoid line breaks. be careful though, the length is normally calculated to avoid any overlap with your personal info; use this at your own typographical risks...
\usepackage{siunitx}
\DeclareSIUnit\year{yr}
\usepackage[version=4]{mhchem}
\usepackage{tabularx}
\setlength{\tabcolsep}{12pt}

\usepackage{ragged2e}

% personal data
\name{Michinari}{Sakai}
\title{Resumé title}                               % optional, remove / comment the line if not wanted
\address{1235 Solano Ave Apt 10}{Albany, CA 94706}{USA}% optional, remove / comment the line if not wanted; the "postcode city" and and "country" arguments can be omitted or provided empty
\phone[mobile]{808-206-4357}                   % optional, remove / comment the line if not wanted
%\phone[fixed]{+2~(345)~678~901}                    % optional, remove / comment the line if not wanted
%\phone[fax]{+3~(456)~789~012}                      % optional, remove / comment the line if not wanted
\email{michsakai@gmail.com}                               % optional, remove / comment the line if not wanted
%\homepage{www.johndoe.com}                         % optional, remove / comment the line if not wanted
%\extrainfo{additional information}                 % optional, remove / comment the line if not wanted
%\photo[64pt][0.4pt]{picture}                       % optional, remove / comment the line if not wanted; '64pt' is the height the picture must be resized to, 0.4pt is the thickness of the frame around it (put it to 0pt for no frame) and 'picture' is the name of the picture file
%\quote{Some quote}                                 % optional, remove / comment the line if not wanted

% to show numerical labels in the bibliography (default is to show no labels); only useful if you make citations in your resume
%\makeatletter
%\renewcommand*{\bibliographyitemlabel}{\@biblabel{\arabic{enumiv}}}
%\makeatother
%\renewcommand*{\bibliographyitemlabel}{[\arabic{enumiv}]}% CONSIDER REPLACING THE ABOVE BY THIS

% bibliography with mutiple entries
%\usepackage{multibib}
%\newcites{book,misc}{{Books},{Others}}
%----------------------------------------------------------------------------------
%            content
%----------------------------------------------------------------------------------
\begin{document}
%-----       letter       ---------------------------------------------------------
% recipient data
%\recipient{SNOLAB}{Creighton Mine \#9\\1039 Regional Road 24\\Lively, ON P3Y 1N2\\Canada}
\recipient{Rare Event Detection group}{Nuclear and Chemical Sciences Division\\Lawrence Livermore National Laboratory}
\date{December 28, 2019}
\opening{Dear Team Hiring Manager,}
%\closing{Best wishes,}
\closing{Thank you for your time,}
%\enclosure[Attached]{curriculum vit\ae{}, research statement, publication list, list of referees}          % use an optional argument to use a string other than "Enclosure", or redefine \enclname
\makelettertitle

\justifying
As a motivated physicist, I am interested in joining your team on the PROSPECT
short-baseline reactor antineutrino experiment as well as to work on developing
new detector technologies in reactor monitoring and neutrino physics.
I would like to apply for the advertised post-doctoral researcher position.

I am currently a post-doctoral researcher at the University of California,
Berkeley with expertise in large-scale data analysis and computational physics
simulations working in the field of particle/nuclear physics.

With regard to my ability to meet the specific requirements of this job:\\
%\noindent\begin{tabularx}{\linewidth}{@{{}\textbullet\enskip}X@{\quad}@{}}
\noindent\begin{tabularx}{\linewidth}{@{{}\textbullet\enskip}X@{\quad}@{}}
	\textbf{Education:} PhD in experimental neutrino physics (2016) with
	emphasis on directional neutrino detection, Geant4 particle/radiation
	transport simulations, and algorithm development for novel particle
	detection technologies.\\
	\textbf{Experience in neutrino physics and scintillation detector design:}
	Graduate work involved developing/applying neutrino detection algorithms in
	the scintillator neutrino detector KamLAND, as well as designing and
	building a portable local detector for testing the light yield of LAB based
	scintillators in strong electric fields. Other scintillator work includes
	testing its stability in extreme environments such as cold temperatures and
	high pressures.\\
	\textbf{Experience with Geant4 and optical tracking simulations:} Extensive
	experience designing Geant4 detector simulation code both from scratch as
	well as collaboratively within a team. Optical tracking simulation
	experience in scintillator detectors KamLAND, mini-TimeCube, as well as
	using VUV (vacuum ultraviolet) wavelength shifters.\\
	\textbf{Experience with the C++ programming language:} 8 years of extensive
	experience with C/C++/ROOT analyzing large physics data sets.\\
	\textbf{Proficient verbal/written communication skills:} Contributed work
	published in various journals. Talks given at institutions such as Los
	Alamos National Laboratory, Sanford Underground Research Facility (SURF),
	Argonne National Laboratory, DNP (Division of Nuclear Physics)
	conferences.\\
	\textbf{Ability to work independently on technical hardware and software
	tasks:} Currently leading local VUV photon wavelength shifting/detection
	hardware project at UC Berkeley, as well as supervising simulation/analysis
	of detector systematics and data.\\
	\textbf{Initiative and interpersonal skills to work in a collaborative team
	environment:} Worked in 3 multinational collaborations in Japan, Italy, and
	US. Original work successfully interfaced with larger collaboration.
	Mentored PhD-level students in weekly Geant4 tutorials and increased
	productivity by organizing group initiatives.\\
\end{tabularx}

I would appreciate an opportunity to meet and discuss my application at an
interview. I have also sent my resume for your consideration. Please feel free
to let me know if you have any questions.

\makeletterclosing

\end{document}


%% end of file `template.tex'.
