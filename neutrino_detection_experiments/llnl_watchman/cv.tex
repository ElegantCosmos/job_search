%%%%%%%%%%%%%%%%%%%%%%%%%%%%%%%%%%%%%%%%%
% Plasmati Graduate CV
% LaTeX Template
% Version 1.0 (24/3/13)
%
% This template has been downloaded from:
% http://www.LaTeXTemplates.com
%
% Original author:
% Alessandro Plasmati (alessandro.plasmati@gmail.com)
%
% License:
% CC BY-NC-SA 3.0 (http://creativecommons.org/licenses/by-nc-sa/3.0/)
%
% Important note:
% This template needs to be compiled with XeLaTeX.
% The main document font is called Fontin and can be downloaded for free
% from here: http://www.exljbris.com/fontin.html
%
%%%%%%%%%%%%%%%%%%%%%%%%%%%%%%%%%%%%%%%%%

%----------------------------------------------------------------------------------------
%	PACKAGES AND OTHER DOCUMENT CONFIGURATIONS
%----------------------------------------------------------------------------------------

\documentclass[10pt]{article} % Default font size and paper size

\usepackage{fontspec} % For loading fonts
%\defaultfontfeatures{Mapping=tex-text}
\setmainfont[SmallCapsFont = Fontin SmallCaps]{Fontin} % Main document font

%\usepackage{xunicode,xltxtra,url} % Formatting packages

\usepackage[usenames,dvipsnames]{xcolor} % Required for specifying custom colors

%\usepackage[big]{layaureo} % Margin formatting of the A4 page, an alternative to layaureo can be \usepackage{fullpage}
% To reduce the height of the top margin uncomment: \addtolength{\voffset}{-1.3cm}
\usepackage[margin=0.7in]{geometry}
%\usepackage[margin=1.27cm]{geometry} % probably too narrow

\usepackage{hyperref} % Required for adding links	and customizing them
\definecolor{linkcolour}{rgb}{0,0.2,0.6} % Link color
\hypersetup{colorlinks,breaklinks,urlcolor=linkcolour,linkcolor=linkcolour} % Set link colors throughout the document

\usepackage{titlesec} % Used to customize the \section command
\usepackage{amssymb}
\usepackage{siunitx}
\usepackage[version=4]{mhchem}
\usepackage{bibentry}
\usepackage{tabularx}
\newcolumntype{R}{>{\raggedleft\arraybackslash}X}  % right multiline alignment
\usepackage{booktabs}% http://ctan.org/pkg/booktabs
\newcommand{\bulletitem}{~~\llap{\textbullet}~~}
\newcommand{\tabitem}{~~\llap{}~~}
\usepackage{enumitem}
\setlist{nosep,after=\strut}
\usepackage{amsmath,amssymb}

%\titleformat{\section}{\Large\scshape\centering}{}{0em}{}[\titlerule] % Text formatting of sections
\titleformat{\section}{\Large\scshape}{}{0em}{}[\titlerule] % Text formatting of sections
\titlespacing{\section}{0pt}{-3pt}{-3pt} % Spacing around sections

\begin{document}

\setlength{\parskip}{\baselineskip}%
\pagestyle{empty} % Removes page numbering

%\font\fb=''[cmr10]'' % Change the font of the \LaTeX command under the skills section

%----------------------------------------------------------------------------------------
%	NAME AND CONTACT INFORMATION
%----------------------------------------------------------------------------------------

%\begin{center}
	\noindent{\huge{\textsc{Michinari Sakai}}}\\% Your name
	\noindent\begin{tabular}{r @{$\quad\bullet\quad$} l}
		\href{mailto:michsakai@ucla.edu}{michsakai@ucla.edu} & 808-206-4357
	\end{tabular}
%\end{center}

%\section{Personal Data}
%
%\begin{center}
%\noindent\begin{tabular}{rl}
%%\textsc{Place and Date of Birth:} & Los Angeles, USA  | 16 October 1980\\
%%\textsc{Place of Birth:} & Los Angeles, USA\\
%%\textsc{Address:} & 475 Portola Plaza \#5-123B, Los Angeles, CA 90095, USA\\
%%\textsc{Phone:} & +1-808-206-4357\\
%%\textsc{email:} & \href{mailto:michsakai@ucla.edu}{michsakai@ucla.edu}
%\end{tabular}
%\end{center}

%----------------------------------------------------------------------------------------
%	SCHOLARSHIPS AND AWARDS
%----------------------------------------------------------------------------------------

%\section{Scholarships and Awards}
%
%\noindent\begin{tabular}{rl}
%	2004 & Award for Outstanding Academic Achievement, Samsung Corp.\\
%	2001 - 2004 & Undergraduate Achievement Scholarships, Sun Moon
%	Univ.\\
%	2001 & Ae-Guk Freshman Scholarship, Sun Moon Univ.\\
%\end{tabular}

%----------------------------------------------------------------------------------------
%	SUMMARY
%----------------------------------------------------------------------------------------

\section{Summary}
\noindent\begin{tabularx}{\linewidth}{@{{}\textbullet\enskip}X@{\quad}r@{}}
	\addlinespace[5pt]
	Expertise in \textsc{Geant4} and particle simulations with 8 years
	of experience & \\
	Experience with cosmogenic radiation and nuclear decay simulations in 3
	major particle detector experiments & \\
	Innovative problem solving skills with the ability to interface original
	work with larger collaboration & \\
\end{tabularx}

%----------------------------------------------------------------------------------------
%	RESEARCH EXPERIENCE 
%----------------------------------------------------------------------------------------

%\section{Research Experience}
\section{Experience}

%\noindent\begin{tabular}{r|p{10.4cm}}
\noindent\begin{tabularx}{\linewidth}{@{}lR@{}}
	\addlinespace[5pt]
	%\textbf{CUORE (Cryogenic Underground Observatory for Rare Events)} & \textsc{Mar. 15, 2016 -} \textit{Current}\\
	\textsc{UCLA - }Post-doctoral Scholar & \textsc{Mar 2016 -} \textit{Current}\\
	\multicolumn{2}{@{}l@{}}{\begin{minipage}[t]{\linewidth}
		%\textit{Post-doctoral Scholar, University of California, Los Angeles (UCLA)}
		%\textit{University of California, Los Angeles (UCLA)}
		\begin{itemize}
			\item Mentored and worked with 2 undergraduate students to simulate
				radiation shielding structures in \textsc{Geant4} to mitigate
				$\gamma$/$\beta$ backgrounds for next generation
				$0\nu\beta\beta$ searches requiring ultra-low backgrounds
			\item Currently spearheading the development of a precision
				$\alpha$ background model with goal for further background
				reduction to reach sensitivity goal of covering the inverted
				neutrino mass hierarchy of $0\nu\beta\beta$ decay in
				\ce{^{130}Te}
			%\item Implemented bolometer thermal model to create mock data for
			%	first data taking
		\end{itemize}
	\end{minipage}}\\

	\addlinespace[5pt]
	%\textsc{Mini-TimeCube (World's Smallest Portable Neutrino Detector)} & \textsc{2009 - 2016}\\
	\textsc{University of Hawaii at Manoa - }Research Assistant & \textsc{2009 - 2016}\\
	\multicolumn{2}{@{}l@{}}{\begin{minipage}[t]{\linewidth}
		%\textit{Research Assistant, University of Hawaii at Manoa}
		%\textit{University of Hawaii at Manoa}
		\begin{itemize}
			\item Led development of \textsc{Geant4} detector simulation to
				conducted case studies of neutron capture doping agents in
				solid scintillator. Simulation results were later used to
				oversee design during detector construction
			%\item Led development of \textsc{Geant4} detector simulation with
			%	team of 3 undergraduate students to conducted case studies of
			%	neutron capture doping agents in solid scintillator. Simulation
			%	results were later used to guide overall detector design during
			%	construction
			%\item Led critical role in directional neutrino/neutron
			%	reconstruction algorithm
			\item Was responsible for background simulation studies associated
				with long lived cosmogenic isotopes \ce{^{8}He}/\ce{^{9}Li} to
				quantitatively determine effect on detector live time
			%\item Spearheaded trade studies for \ce{Li}/\ce{B} neutron capture
			%	dopants in scintillator
	%	\end{itemize}
	%\end{minipage}}\\
	%\textsc{KamLAND (Kamioka Liquid Scintillator Antineutrino Detector)} & \textsc{2009 - 2016}\\
	%%Research Assistant, University of Hawaii & \textsc{Aug 20, 2009 - May 14, 2016}\\
	%\multicolumn{2}{@{}l@{}}{\begin{minipage}[t]{\linewidth}
	%	%\textit{Research Assistant, University of Hawaii at Manoa}
	%	\textit{University of Hawaii at Manoa}
	%	\begin{itemize}
			\item Was responsible for high energy ($\gtrsim
				\SI{1}{\giga\electronvolt}$) energy calibration using cosmic
				ray muons and applying this to neutrino analysis for first time
			\item Spearheaded development of novel directional neutrino
				detection technique in scintillator and demonstrated with data
				for the first time that this can be applied to conduct indirect
				dark matter searches in scintillator. First ever physics
				application of neutrino directionaly in scintillator
			\item Led unprecedented particle ID capability studies in
				scintillator using track profile reconstruction techniques
				using never before observed T2K events spilling into KamLAND
		\end{itemize}
	\end{minipage}}\\
\end{tabularx}

%----------------------------------------------------------------------------------------
%	SKILLS 
%----------------------------------------------------------------------------------------

\section{Skills}

\noindent\begin{tabularx}{\linewidth}{@{}rl}
	\addlinespace[5pt]
	Software/Tools: & \textsc{Geant4}, \textsc{ROOT}, \textsc{Pads}, \textsc{AutoCAD}\\
	Programming Languages: & Proficient in C, C++, Python, Fortran, Mathematica, BASH\\
	Human Languages: & English (native), Japanese/Korean (trilingual proficiency)\\
\end{tabularx}

%----------------------------------------------------------------------------------------
%	TEACHING
%----------------------------------------------------------------------------------------

%\section{Leadership and Teaching Experience}
\section{Leadership}

\noindent\begin{tabularx}{\linewidth}{@{}lR@{}}
	\addlinespace[5pt]
	\textsc{Mentor}, \textit{UCLA} & \textsc{Mar 2016 -} \textit{Current}\\
	\multicolumn{2}{@{}l@{}}{\begin{minipage}[t]{\linewidth}
		\begin{itemize}
			\item Taught weekly \textsc{Geant4} tutorials to \num{3} PhD-level
				students and an undergraduate student for 1 semester;
				students are now able to take on simulation projects of their
				own and make original contribution
			%\item Led weekly Physics paper discussion groups for \num{3} PhD
			%	students, and promoted team work to increase dialogue and
			%	productivity within team
		\end{itemize}
	\end{minipage}}\\

	\addlinespace[5pt]
	%\textsc{Aug 2007 - May 2009} & Teaching Assistant, University of Hawaii at Manoa\\
	\textsc{Teaching Assistant}, \textit{University of Hawaii at Manoa} & \textsc{2007 - 2009}\\
	\multicolumn{2}{@{}l@{}}{\begin{minipage}[t]{\linewidth}
		\begin{itemize}
			\item Planned classwork and taught 2 weekly undergraduate Physics
				Laboratory classes of over \num{20} students each for 3
				semesters, received very positive reviews
			\item Mentored undergraduate students in undergraduate Physics
				classwork for 2 hours each week for 3 semesters
		\end{itemize}
	\end{minipage}}\\

%------------------------------------------------

%%\textsc{Jan 2003 - Mar 2006} & Interpreter and Teacher\\
%	\textbf{Interpreter and Teacher} & \textsc{2003 - 2006}\\
%	\multicolumn{2}{X}{Part time English lecturer for Korean undergraduate students.}\\
%	\multicolumn{2}{X}{Part time contributing reporter and translator for campus magazine.}\\
%	\multicolumn{2}{X}{Spontaneous trilingual interpreter for W-CARP International Education Conference.}\\
%	\multicolumn{2}{X}{Part time translator for magazine Today's World.}\\
\end{tabularx}

%----------------------------------------------------------------------------------------
%	EDUCATION
%----------------------------------------------------------------------------------------

\section{Education}

%\noindent\begin{tabular}{rp{10.3cm}}	
	\noindent\begin{tabularx}{\linewidth}{@{}lR@{}}	
	\addlinespace[5pt]
	%\textsc{PhD, Experimental Neutrino Physics} & \textsc{Aug. 20, 2007 - May 14, 2016}\\
	%\textsc{PhD, Experimental Neutrino Physics} & \textsc{2016}\\
	\textsc{PhD, Experimental Particle Physics} & \textsc{2016}\\
	\multicolumn{2}{@{}l@{}}{\begin{minipage}[t]{\linewidth}
		\begin{itemize}
			\item[] GPA: 4.0/4.0, University of Hawaii at Manoa
			\item[] Dissertation: High Energy Neutrino Analysis at KamLAND and
				Application to Dark Matter Search
		\end{itemize}
	\end{minipage}}\\
	%\small GPA: 3.97/4.00 &\\
	%\small Advisor: Prof. John G. \textsc{Learned} & \\

%------------------------------------------------

	%\textsc{Graduate Program in Mathematics} & \textsc{2006}\\
	%\multicolumn{2}{@{}l@{}}{\begin{minipage}[t]{\linewidth}
	%	\begin{itemize}
	%		\item[] GPA: 4.5/4.5, Sun Moon University, S.~Korea
	%	\end{itemize}
	%\end{minipage}}\\
	%%\small GPA: 4.50/4.50 &\\
	%%& \small Advisor: Prof. Doe-Wan \textsc{Kim}\\

%------------------------------------------------

	\addlinespace[5pt]
	\textsc{Double BS, Physics and Mathematics} & \textsc{2005}\\
	\multicolumn{2}{@{}l@{}}{\begin{minipage}[t]{\linewidth}
		\begin{itemize}
			\item[] GPA: 4.3/4.5, Sun Moon University, S.~Korea
			\item[] President's Award 2005, Award for Outstanding Academic Achievement -- Samsung Corp.
		\end{itemize}
	\end{minipage}}\\
	%\small Summa Cum Laude, GPA: 4.33/4.50&\\
	%& \small Advisor: Prof. Ki-Won \textsc{Kim}\\

%------------------------------------------------

\end{tabularx}

%----------------------------------------------------------------------------------------
%	LANGUAGES
%----------------------------------------------------------------------------------------

%\section{Languages}
%
%\noindent\begin{tabular}{rl}
%	\textsc{English}, \textsc{Japanese}, \textsc{Korean}\\
%\end{tabular}
%
%\newpage

%----------------------------------------------------------------------------------------
%	POSTERS AND TALKS
%----------------------------------------------------------------------------------------

\clearpage
\section{Talks and Presentations}

\noindent\begin{tabularx}{\linewidth}{@{{}\textbullet\enskip}X@{\quad}r@{}}
\addlinespace[5pt]
Invited Talk (Tentative): \textsc{Monte Carlo Tools in CUORE} \newline Durham University, UK - Monte Carlo Tools for Beyond the Standard Model Physics & Apr 2018 \\

\addlinespace[5pt]
Seminar: \textsc{CUORE: A Bolometric Search for Lepton Number Violation} \newline Argonne National Laboratory & Feb 2018 \\

\addlinespace[5pt]
Talk: \textsc{CUORE and Background Reduction Case Studies for CUPID} \newline Pittsburgh/Carnegie Mellon University - Division of Nuclear Physics 2017 & Oct 2017 \\

\addlinespace[5pt]
Invited talk: \textsc{Status of the CUORE $0\nu\beta\beta$ Decay Search} \newline Sanford Underground Research Facility (SURF), South Dakota - Conference on Science at SURF 2017 & May 2017 \\

\addlinespace[5pt]
Invited talk: \textsc{Particle ID and event reconstruction algorithms in scintillator} \newline Fermilab - Frontiers of Liquid Scintillator Technology & Mar 2016 \\

\addlinespace[5pt]
Seminar: \textsc{High Energy Analysis at KamLAND and Application to Dark Matter Search} \newline Los Alamos National Laboratory & Nov 2015 \\

\addlinespace[5pt]
Seminar: \textsc{High Energy Analysis at KamLAND and Application to Dark Matter Search} \newline California Institute of Technology & Nov 2015 \\

\addlinespace[5pt]
Seminar: \textsc{High Energy Analysis at KamLAND and Application to Dark Matter Search} \newline University of California, Los Angeles & Oct 2015 \\

\addlinespace[5pt]
Talk: \textsc{High Energy Analysis and Application to Dark Matter Search in KamLAND} \newline University of Hawaii at Manoa - DOE project review & Jul 2015 \\

\addlinespace[5pt]
Poster: \textsc{Indirect Dark-Matter Detection Through KamLAND} \newline Neutrino 2012, Kyoto, Japan & Jun 2012 \\

\addlinespace[5pt]
Talks: \textsc{What is a Neutrino?}, \textsc{mini-TimeCube: The World's Smallest Neutrino Detector} \newline University of Hawaii at Manoa - Campus Open-house & Nov 2010/2011 \\

\addlinespace[5pt]
Talk: \textsc{mini-TimeCube: A Portable Directional Neutrino Detector} \newline Applied Antineutrino Physics 2010, Sendai, Japan & Aug 2010 \\

\addlinespace[5pt]
Talk: \textsc{KamLAND Summary} \newline University of Hawaii at Manoa - DOE project review & Sep 2009 \\

\addlinespace[5pt]
Talk (Student Presentation): \textsc{How to solve $\theta_{23}$ degeneracy} \newline Fermilab - International Neutrino Summer School 2009 & Jul 2009 \\

\end{tabularx}

%----------------------------------------------------------------------------------------
%	PUBLICATIONS
%----------------------------------------------------------------------------------------

\clearpage
\renewcommand\refname{Publications} %changes default name to Publications

\nocite{*} %lists everything in the .bib file
%\bibliographystyle{plain} %hundreds of styles to choose from
\bibliographystyle{ieeetr} %hundreds of styles to choose from
\bibliography{publications.bib} %name of your .bib file
%\newrefcontext[sorting=nty]
%\printbibliography[title=Alphabetic]
%\bibliographystyle{habbrvyr}
%\bibliography{publications.bib} %name of your .bib file


%\section{Publications}
%
%\noindent\begin{tabular}{rp{11cm}}
%	\multicolumn{2}{l}{\textsc{mini-TimeCube}} \\
%	2015 (expected) & V.A. Li et al., \textsc{mini-TimeCube}, RSI Invited Review\\
%	\multicolumn{2}{c}{} \\
%	\multicolumn{2}{l}{\textsc{KamLAND}} \\
%	2015 (expected) & K. Asakura et al., \textsc{Search for the Proton Decay
%Mode $p \rightarrow \overline{\nu} K^{+}$ with KamLAND}, Phys. Rev. D\\
%	Mar. 2015 & K. Asakura et al., \textsc{Study of electron anti-neutrinos
%	associated with gamma-ray bursts using KamLAND}, arXiv:1503.02137v1\\
%	Feb. 2015 & T.I. Banks et al., \textsc{A compact ultra-clean system for
%	deploying radioactive sources inside the KamLAND detector},
%	10.1016/j.nima.2014.09.068\\
%	Jan. 2015 & C. Lane et al., \textsc{A new type of Neutrino Detector for
%	Sterile Neutrino Search at Nuclear Reactors and Nuclear Nonproliferation
%	Applications}, arXiv:1501.06935v1\\
%	May 2014 & A. Gando et al., \textsc{7Be Solar Neutrino Measurement with
%	KamLAND}, arXiv:1405.6190v1\\
%	Aug. 2011 & S. Abe et al., \textsc{Measurement of the 8B Solar Neutrino
%	Flux with the KamLAND Liquid Scintillator Detector},
%	10.1103/PhysRevC.84.035804\\
%	Aug. 2011 & J. Kumar, J.G. Learned, M. Sakai, S. Smith,
%	\textsc{Dark Matter Detection With Electron Neutrinos in Liquid
%	Scintillation Detectors}, Phys. Rev. D84 (2011) 036007\\
%\end{tabular}

%----------------------------------------------------------------------------------------
%	STATEMENT OF RESEARCH
%----------------------------------------------------------------------------------------

\clearpage
\section{Statement of Research}

In have a strong background in radiation/particle interaction simulations and
how they affect real particle detector hardware. My expertise also
involves processing and analyzing large physics data sets using analysis
tools such as ROOT and Python. I have experience working in 3 multinational
collaboration experiments where I spearheaded independent research initiatives
and successfully interfaced my original work with the larger team. Some of my
past accomplishments that I take pride in include development of unprecedented
particle detection algorithms that I project will open whole new physics
searches by the experiments that I collaborated with.

I have worked as the lead \textsc{Geant4} simulation designer for the
mini-TimeCube experiment at University of Hawaii at Manoa from August 2009 to
March 2016. Mini-TimeCube is an ambitious project to build the world's smallest
portable neutrino detector. In this project, I mentored 3 undergraduate
students and worked in collaboration with them to conduct case studies for
optimizing the detector design, test candidate neutron capture doping elements
in plastic scintillator, and simulate the response of the multi-channel-plate
(MCP) photomultiplier tubes (PMTs) deployed in the detector. The studies were
used during construction of the detector, and to develop directional particle
detection algorithms that are now being tested in analyses of neutrons from
test sources as well as neutrinos from nuclear reactors at the National
Institute of Standards and Technology (NIST). I have also conducted simulation
studies for cosmic-ray muons and long-lived cosmogenic background isotopes such
as \ce{^{8}He} and \ce{^{9}Li}. These backgrounds are extremely difficult to
tag due to their long life-time ($\gtrsim \si{\second}$ scale) and travel
distances. The studies have been vital to the project.
%Working with the
%mini-TimeCube project has further involved designing and fabricating PCB boards
%as well as contributing to the FPGA firmware for the readout electronics.
A paper summarizing our accomplishments was published in 2016 (V. A. Li et al.
Invited Article: miniTimeCube. Rev. Sci. Instrum., 87(2):021301, 2016,
1602.01405).

I have been involved with the CUORE (Cryogenic Underground Observatory for Rare
Events) experiment at the University of California, Los Angeles since March of
2016. The main objective of the CUORE experiment is to hunt for lepton number
violation by observing neutrinoless double beta ($0\nu\beta\beta$) decay in
\ce{^{130}Te}. CUORE employs an almost 20 fold increase in detector mass
compared to its previously successful pilot experiment CUORE-0. My work in the
collaboration currently involves development of a \textsc{Geant4} based
precision background model together with a graduate student colleague to better
understand the radioactive contaminations in the detector. The energy spectrum
of the backgrounds in the so called $\alpha$ region ($\gtrsim
\SI{2.7}{\mega\electronvolt}$) exhibit peculiar features that, if understood
correctly, will better explain the types of contamination sources and their
distributions in the materials comprising the experiment. This can help us to
better understand our backgrounds and extrapolate this new knowledge to the
energy region of interest (\SIrange{2465}{2575}{\kilo\electronvolt}) for
$0\nu\beta\beta$ decay in \ce{^{130}Te}. I have previously also mentored 2
undergraduate students and worked together with them to simulate and
investigate new radioactivity shielding schemes for further background
reduction in future $0\nu\beta\beta$ decay experiments that will cover the
inverted hierarchy region of the effective Majorana neutrino mass. A paper for
our first $0\nu\beta\beta$ analysis using CUORE data was published in March
2018 (https://arxiv.org/abs/1710.07988).

While working at the University of Hawaii at Manoa, I also developed a novel
directional event reconstruction algorithm for high-energy
$\gtrsim$\si{\giga\electronvolt} scale neutrinos while working with KamLAND
(Kamioka Liquid Scintillation Antineutrino Detector), and demonstrated with
data that this technique can be applied to indirect dark matter search by
looking for a directional flux of neutrinos from the core of the Sun and Earth.
Studies done with Monte Carlo suggest that the accuracy of deducing the
neutrino direction using this new method is better than that of water-Cherenkov
detectors (the conventional method for directional neutrino detection) by
$\sim$\SI{10}{\degree} in this energy regime. This method was verified using
never before observed neutrino events spilling into KamLAND from the T2K
neutrino beam-line. The results were consistent with expectation. According to
my knowledge, this is the first ever physics application of neutrino
directionality in scintillator.

%My work with KamLAND further involved demonstration of 3-dimensional
%topological event imaging techniques, originally developed in the LENA (Low
%Energy Neutrino Astronomy) collaboration, using data for the first time. The
%$\sim$\SI{3.5}{\nano\second} timing resolution of the PMTs (photomultiplier
%tubes) employed in KamLAND are not good enough to do a detailed imaging of all
%the individual tracks in a neutrino event. Nevertheless
%$\gtrsim$\si{\giga\electronvolt} muon tracks and high enough energy tracks in a
%neutrino event were imaged as well as the overall direction of the final state
%particles to resolve the incoming neutrino direction. In addition
%$\frac{\mathrm{d}E}{\mathrm{d}x}$ profiles were investigated to perform
%unprecedented particle ID studies in scintillator at these energies. A paper
%employing these techiques I developed to conduct an indirect dark matter search
%is currently under preparation.

%In order to promote more dialogue and efficient teamwork, I have also organized
%and led weekly Physics paper discussion group sessions involving both
%undergraduate and graduate students in our laboratory at UCLA. These sessions
%have helped us not only to know our own experiment better, but to increase our
%common understanding of the overall field of experimental $0\nu\beta\beta$
%decay search. Ultimately this has been invaluable for our team to increase
%our teamwork and productivity, and become a valuable contributor to the collaboration.
%A paper for our first $0\nu\beta\beta$ analysis using CUORE data
%was submitted for publication to PRL in late 2017, and is currently under
%review (https://arxiv.org/abs/1710.07988).

%The focus of my work at the University of Hawaii has been in the directional
%detection of high energy neutrinos in scintillator and its application toward
%indirect dark matter searches, directional geo-neutrino measurements, and
%anti-nuclear proliferation techniques that involve locating the position of the
%neutrino source.
%My main interest has been in detector development and novel techniques to
%search for previously unexplored phase-space regions in neutrino physics. My
%main interest and past research experience has been in detector development
%and developing novel techniques to open new methods of exploration in neutrino
%physics. 

%In addition, I have also worked on 3 other projects: the
%\SI{1}{\kilo\tonne} liquid scintillator neutrino detection experiment KamLAND
%(Kamioka Liquid Scintillator Antineutrino Detector) in Kamioka, Japan, a
%portable \SI{2.2}{\liter} plastic scintillator neutrino detector called the
%mini-TimeCube, and scintillator R\&D for a future \SI{10}{\kilo\tonne}-scale
%deep-sea based neutrino detector HanoHano.
%
%My work in KamLAND has involved developing directional event reconstruction
%methods for high-energy $\gtrsim$\si{\giga\electronvolt} scale neutrinos and
%applying this to conduct an indirect dark matter search by looking at an excess
%flux of neutrinos from the core of the Sun and Earth. Studies done with Monte
%Carlo suggest that the accuracy of reconstructing the neutrino direction using
%this method is better than that of water-Cherenkov detectors by
%$\sim$\SI{10}{\degree} for this energy regime. This method was tested against
%initial events spilling into KamLAND from the T2K neutrino beam-line and the
%results were consistent with what was expected. I believe this is the first
%ever physics application of neutrino directionality in a scintillator
%experiment.

%Finally, my work in scintillator R\&D for HanoHano, a proposed
%\SI{10}{\kilo\tonne}-scale deep-sea based neutrino detector, involved designing
%and building apparatuses using CAD for measuring light output of Linear
%alkylbenzene (LAB) based liquid scintillators when put in large electric
%potential gradients as well as testing their light transmissivities under
%extreme temperatures and pressures such as those found in deep-sea
%environments. This project included mentoring an undergraduate student on
%techniques for shielding electronic apparatuses and working with another
%graduate student on designing and operating the cold high pressure environment
%device.

In summary my strong background in radiation/particle interaction simulations
and how they affect real particle detectors, as well as, my comfort in using
tools to analyze large physics data sets makes me a strong candidate for your
position. Also I would like to reemphasize my previous accomplishments in
independently creating novel particle detection techniques to solve complex
problems and interface my original work with a large team of collaborators. I
believe I can make a significant impact in your team.

%%----------------------------------------------------------------------------------------
%%   STATEMENT OF TEACHING PHILOSOPHY AND EXPERIENCE
%%----------------------------------------------------------------------------------------
%
%\clearpage
%\section{Statement of Teaching Philosophy and Experience}
%
%Throughout my academic career, I have been heavily involved in teaching and
%mentoring students from a wide range of cultures and backgrounds. This has
%included working as a teaching assistant during my graduate studies at
%University of Hawaii at Manoa where I organized and led 2 undergraduate Physics
%Laboratory classes of over 20 students each for a total of 3 semesters. In
%addition, I have also led weekly, 2 hour long, one-on-one tutoring sessions for
%3 semesters. Student evaluations for the effectiveness of my teaching and
%ability to communicate information were ``excellent''.
%
%My philosophy for teaching can be summarized in two points: self motivated
%drive through curiosity, and mastering through iteration. Fist, when teaching
%at any level, and especially at the undergraduate level, it is very important
%to never get bogged down by the equation or numbers. When I begin a class or
%introduce a new concept to students, I always first demonstrate the experiment
%itself or introduce the context in an illustrative way. This is to engage the
%students’ interest and entice their curiosity to develop a long lasting self
%motivated drive to learn. Curiosity is where science is born from and curiosity
%is what drives science forward. This is true whether one is an freshman student
%or a Nobel laureate. Second, iteration is the key to mastering a topic. No
%matter how gifted one may be, no one is able to perfectly master a topic
%instantaneously in a single trial. On the other hand, this also means that even
%if one is not naturally gifted at something, it can be mastered through
%repetitive iteration and failure. The process of iterative learning is how the
%human brain naturally works and is the key to effective learning.
%
%As an example of the above, there was a Physics Laboratory class in which I
%needed to introduce damping of a driven harmonic oscillator in a medium with
%finite viscosity. I knew that the equations describing the physics behind this
%process could be tedious and difficult to grasp at the freshman level I was
%teaching at. But I also knew that the concepts themselves were not inherently
%hard because we all intuitively know from our experiences swimming in a pool,
%how viscous media affects forces. Before going through the equations, I turned
%on the experimental apparatus first, and asked the students what they thought
%would happen when the frequency of the driving force was modified in different
%ways, while at the same time, observing their satisfaction and surprise when
%their predictions were confirmed or debunked. This effectively created a vastly
%more engaged and thoughtful atmosphere for the class and let the students
%concentrate on the ``physics'' of the class while helping them to more
%intuitively understand the equations involved. Of course, it is equally
%important for the instructor to not ``give away'' all the answers by doing all
%the work. The students were left to make multiple failed attempts until they
%were able to get just the right settings to take successful measurements.
%
%In society today, there is an unfortunate misconception that Physics is a
%difficult subject only for \textit{smart} students. Nothing could be further
%from the truth. When students feel that they are not smart enough, it is
%important to let them know that Physics is not about intelligence, but about
%curiosity driven iterative learning. All professional Physicists stand on the
%shoulders of past giants, and almost no research today is conducted solely
%alone, but through collaboration and teamwork. No one is \textit{smart} enough
%to do all the work alone. I usually use myself as an example. I come from a
%small regional college in South Korea with very little opportunity, but through
%diligence and iterative trial, I have had the opportunity to eventually work
%with some of the brightest minds in some of the most innovative projects in the
%field of neutrino physics.
%
%Mentoring higher level graduate students may be slightly different but I
%believe the same overall approach remains the same. I have taught \textsc{Geant4} Monte
%Carlo particle simulations to 3 undergraduate and 3 Ph.D. students at the
%University of California, Los Angeles (UCLA). My method for teaching this
%toolkit is to learn together with the students by going into the details of the
%code and doing this repetitively every week. Not all the concepts may be clear
%to students in just the first trial, but through iteratively going through and
%encountering similar problems multiple times, students will eventually
%understand and learn basic coding techniques as well as tricky concepts that
%may not have been evident at first. Here, it is important to be immersed in the
%task along with the students instead of teaching from a conventional top-down
%approach because learning how to code is not just about acquiring the skill to
%code, but about learning the culture of coding. By this I mean that there are
%certain useful tools and methods of accomplishing a task that can only be
%learned through going into gritty details and getting \textit{dirty}. As
%students grow and become more knowledgeable, they can start to develop and
%choose their own methods and tools. Through this approach, my students have
%successfully been able to take on simulation tasks of their own and contribute
%effectively to the collaboration.

%%----------------------------------------------------------------------------------------
%%   STATEMENT OF CONTRIBUTIONS TO DIVERSITY
%%----------------------------------------------------------------------------------------
%\clearpage
%\section{Statement of Contribution to Diversity}
%
%Having spent 15 years of my adolescent life in South Korea, and being fully
%Japanese by descent, while being born in the United States, I have had the
%unique opportunity to develop an \emph{international} perspective of things
%from a very young age. My ability to converse fluently in 3 languages has also
%greatly helped me to view different people and situations from multiple angles.
%With this sort of upbringing, I have learned from an early age that not only is
%diversity important, but tolerant diversity is just as important. Somehow it is
%the human tendency in a diverse environment, whether it may be in terms of
%race, culture, language, or ideas, when proper communication cannot be made,
%for people to lump the ``other group'' into one generic category. This can be
%for better or for worse, but often times manifests itself in unproductive ways
%when team work is required to reach a common goal. Due to my background and
%abilities, I have countless times served as a mediator of sorts between
%opposing groups often times as an interpreter, translator, or just simply as a
%arbitrator.
%
%This way of thinking has served me invaluably also in my professional career to
%put myself in other peoples' shoes. I have contributed to the education of a
%diverse body of students through Physics Laboratory courses that I taught at
%University of Hawaii at Manoa for 3 semesters from 2007 to 2009. The ethnic and
%cultural atmosphere in Hawaii is one of the most diverse on the planet. The
%University of Hawaii system not only attracts students internationally from
%Asia and from around the globe, but also serves as the hub of higher learning
%for minority students coming from nations in the Pacific Rim such as the
%Marshall Islands and Micronesia. In this cultural melting pot, instructors are
%required to effectively convey information in an inclusive and efficient way
%while understanding each of the students' cultures and needs as best as
%possible. For example, these needs may include supplemental instruction if
%English is not the student's primary language, or specific attention if the
%student comes from a background where STEM is not traditionally emphasized.
%
%In addition, my research has allowed me to live and work in various countries
%around the globe such as Japan and Italy giving me the opportunity to work with
%a wide range of ethnicities and cultures. For example, during my time in Japan,
%I volunteered to be a presenter to introduce and explain about the various
%research activities conducted by our neutrino experiment (KamLAND) to Japanese
%undergraduate students at Tohoku University. Through this event, I was able
%appeal to diversity by conveying the importance of international collaboration
%that our experiment requires to successfully operate the experiment.
%
%I believe that my multi-cultural upbringing and experiences working in very
%diverse teams of international collaborators, can significantly contribute to
%the spectrum of diversity and inclusive excellence that UCI cherishes.

%----------------------------------------------------------------------------------------
%	REFERENCES
%----------------------------------------------------------------------------------------

\clearpage
\section{References}

\noindent\begin{tabular}{rp{12cm}}
\multicolumn{2}{l}{\footnotesize{Supplied upon request or please contact in
	person.}}\\
\multicolumn{2}{c}{}\\

Huan Z. \textsc{Huang} & Professor, University of California, Los Angeles, +1-310-825-9297\\
& \href{mailto:huang@physics.ucla.edu}{\nolinkurl{huang@physics.ucla.edu}}\\
& 475 Portola Plaza \#5-136, Los Angeles, CA 90095-1547, USA\\
\\
John G. \textsc{Learned} & Professor, University of Hawaii at Manoa, +1-808-956-2964\\
& \href{mailto:jgl@phys.hawaii.edu}{\nolinkurl{jgl@phys.hawaii.edu}}\\
& 2505 Correa Rd. \#327, Honolulu, Hawaii 96822, USA\\
\\
Yury \textsc{Kolomensky} & Professor, University of California, Berkeley, +1-510-642-9619\\
& \href{mailto:ygkolomensky@lbl.gov}{\nolinkurl{ygkolomensky@lbl.gov}}\\
& LeConte Hall \#319, Berkeley, CA, 94720-7300, USA\\
\\
Brian K. \textsc{Fujikawa} & Staff Scientist, Lawrence Berkeley National Laboratory, +1-510-486-4398\\
& \href{mailto:bkfujikawa@lbl.gov}{\nolinkurl{bkfujikawa@lbl.gov}}\\
& 1 Cyclotron Rd MS 50R5008, Berkeley, CA 94720-8158, USA\\
\\
Lindley \textsc{Winslow} & Jerrold R. Zacharias Assistant Professor, MIT, +1-617-253-2332\\
& \href{mailto:lwinslow@mit.edu}{\nolinkurl{lwinslow@mit.edu}}\\ 
& 77 Massachusetts Avenue, Bldg. 26-569, Cambridge, MA 02139, USA\\
\\
Thomas \textsc{O'Donnell} & Assistant Professor, Virginia Tech, +1-540-231-3308\\
& \href{mailto:tdonnell@vt.edu}{\nolinkurl{tdonnell@vt.edu}}\\
& 850 West Campus Drive \#313, Blacksburg, VA 24061, USA\\

%Kunio \textsc{Inoue} & Professor, Tohoku Univ./RCNS, +81-22-795-6727,
%\href{mailto:inoue@awa.tohoku.ac.jp}{\nolinkurl{inoue@awa.tohoku.ac.jp}}\\
%
%Jason \textsc{Kumar} & Assoc. Professor, Univ. of Hawaii at Manoa, +1-808-956-2972,
%\href{mailto:jkumar@phys.hawaii.edu}{\nolinkurl{jkumar@phys.hawaii.edu}}\\
%
%Jelena \textsc{Maricic} & Assoc. Professor, Univ. of Hawaii at Manoa, +1-808-956-7176,
%\href{mailto:jelena@phys.hawaii.edu}{\nolinkurl{jelena@phys.hawaii.edu}}\\

%Adam \textsc{Bernstein} & P.I. Applied Antineutrino Physics, LLNL,
%\href{mailto:bernstein3@llnl.gov}{\nolinkurl{bernstein3@llnl.gov}}\\
%\multicolumn{2}{l}{\footnotesize{(note: I have met Dr. Bernstein once during AAP
%2010 Sendai, so perhaps he knows me least within the listed referrers.)}}\\
\end{tabular}

%----------------------------------------------------------------------------------------

\end{document}
