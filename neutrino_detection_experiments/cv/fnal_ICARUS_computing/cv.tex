%%%%%%%%%%%%%%%%%%%%%%%%%%%%%%%%%%%%%%%%%
% Plasmati Graduate CV
% LaTeX Template
% Version 1.0 (24/3/13)
%
% This template has been downloaded from:
% http://www.LaTeXTemplates.com
%
% Original author:
% Alessandro Plasmati (alessandro.plasmati@gmail.com)
%
% License:
% CC BY-NC-SA 3.0 (http://creativecommons.org/licenses/by-nc-sa/3.0/)
%
% Important note:
% This template needs to be compiled with XeLaTeX.
% The main document font is called Fontin and can be downloaded for free
% from here: http://www.exljbris.com/fontin.html
%
%%%%%%%%%%%%%%%%%%%%%%%%%%%%%%%%%%%%%%%%%

%----------------------------------------------------------------------------------------
%	PACKAGES AND OTHER DOCUMENT CONFIGURATIONS
%----------------------------------------------------------------------------------------

\documentclass[10pt]{article} % Default font size and paper size

%\usepackage{fontspec} % For loading fonts
%\defaultfontfeatures{Mapping=tex-text}
%\setmainfont[SmallCapsFont = Fontin-SmallCaps]{Fontin} % Main document font

%\usepackage{xunicode,xltxtra,url} % Formatting packages

\usepackage[usenames,dvipsnames]{xcolor} % Required for specifying custom colors

%\usepackage[big]{layaureo} % Margin formatting of the A4 page, an alternative to layaureo can be \usepackage{fullpage}
% To reduce the height of the top margin uncomment: \addtolength{\voffset}{-1.3cm}
\usepackage[margin=0.7in]{geometry}

\usepackage{hyperref} % Required for adding links	and customizing them
\definecolor{linkcolour}{rgb}{0,0.2,0.6} % Link color
\hypersetup{colorlinks,breaklinks,urlcolor=linkcolour,linkcolor=linkcolour} % Set link colors throughout the document

\usepackage{titlesec} % Used to customize the \section command
\usepackage{amssymb}
\usepackage{siunitx}
\usepackage[version=4]{mhchem}
%\usepackage{bibentry}
\usepackage[sorting=ydnt]{biblatex}
\usepackage{tabularx}
\newcolumntype{R}{>{\raggedleft\arraybackslash}X}  % right multiline alignment
\usepackage{booktabs}% http://ctan.org/pkg/booktabs
\newcommand{\bulletitem}{~~\llap{\textbullet}~~}
\newcommand{\tabitem}{~~\llap{}~~}
\usepackage{enumitem}
\setlist{nosep,after=\strut}
\usepackage{amsmath,amssymb}

\titleformat{\section}{\Large\scshape\raggedright}{}{0em}{}[\titlerule] % Text formatting of sections
\titlespacing{\section}{0pt}{-3pt}{-3pt} % Spacing around sections

\begin{document}

\setlength{\parskip}{\baselineskip}%
\pagestyle{empty} % Removes page numbering

%\font\fb=''[cmr10]'' % Change the font of the \LaTeX command under the skills section

%----------------------------------------------------------------------------------------
%	NAME AND CONTACT INFORMATION
%----------------------------------------------------------------------------------------

\begin{center}
	{\huge{\textsc{Michinari Sakai}}}\\% Your name
	\noindent\begin{tabular}{r @{$\quad\bullet\quad$} l}
		\href{mailto:michsakai@ucla.edu}{michsakai@ucla.edu} & 808-206-4357
	\end{tabular}
\end{center}

%\section{Personal Data}
%
%\begin{center}
%\noindent\begin{tabular}{rl}
%%\textsc{Place and Date of Birth:} & Los Angeles, USA  | 16 October 1980\\
%%\textsc{Place of Birth:} & Los Angeles, USA\\
%%\textsc{Address:} & 475 Portola Plaza \#5-123B, Los Angeles, CA 90095, USA\\
%%\textsc{Phone:} & +1-808-206-4357\\
%%\textsc{email:} & \href{mailto:michsakai@ucla.edu}{michsakai@ucla.edu}
%\end{tabular}
%\end{center}

%----------------------------------------------------------------------------------------
%	EDUCATION
%----------------------------------------------------------------------------------------

\section{Education}

%\noindent\begin{tabular}{rp{10.3cm}}	
	\noindent\begin{tabularx}{\linewidth}{@{}lR@{}}	
	%\textsc{PhD, Experimental Neutrino Physics} & \textsc{2016}\\
		\textsc{PhD, Experimental Particle/Neutrino Physics} & \textsc{2016}\\
	\multicolumn{2}{@{}l@{}}{\begin{minipage}[t]{\linewidth}
		\begin{itemize}
			\item[] GPA: 4.0/4.0, University of Hawaii at Manoa
			\item[] Dissertation: High Energy Neutrino Analysis at KamLAND and Application to Dark Matter Search
		\end{itemize}
	\end{minipage}}\\
	%\small GPA: 3.97/4.00 &\\
	%\small Advisor: Prof. John G. \textsc{Learned} & \\

%------------------------------------------------

	\textsc{Graduate Program in Mathematics} & \textsc{2006}\\
	\multicolumn{2}{@{}l@{}}{\begin{minipage}[t]{\linewidth}
		\begin{itemize}
			\item[] GPA: 4.5/4.5, Sun Moon University, S.~Korea
		\end{itemize}
	\end{minipage}}\\
	%\small GPA: 4.50/4.50 &\\
	%& \small Advisor: Prof. Doe-Wan \textsc{Kim}\\

%------------------------------------------------

	\textsc{Double BS, Physics and Mathematics} & \textsc{2005}\\
	\multicolumn{2}{@{}l@{}}{\begin{minipage}[t]{\linewidth}
		\begin{itemize}
			\item[] GPA: 4.3/4.5, Sun Moon University, S.~Korea
			\item[] President's Award 2005, Award for Outstanding Academic Achievement -- Samsung Corp.
		\end{itemize}
	\end{minipage}}\\
	%\small Summa Cum Laude, GPA: 4.33/4.50&\\
	%& \small Advisor: Prof. Ki-Won \textsc{Kim}\\

%------------------------------------------------

\end{tabularx}

%----------------------------------------------------------------------------------------
%	SCHOLARSHIPS AND AWARDS
%----------------------------------------------------------------------------------------

%\section{Scholarships and Awards}
%
%\noindent\begin{tabular}{rl}
%	2004 & Award for Outstanding Academic Achievement, Samsung Corp.\\
%	2001 - 2004 & Undergraduate Achievement Scholarships, Sun Moon
%	Univ.\\
%	2001 & Ae-Guk Freshman Scholarship, Sun Moon Univ.\\
%\end{tabular}

%----------------------------------------------------------------------------------------
%	RESEARCH EXPERIENCE 
%----------------------------------------------------------------------------------------

\section{Research Experience}

%\noindent\begin{tabular}{r|p{10.4cm}}
\noindent\begin{tabularx}{\linewidth}{@{}lR@{}}
	\textsc{KamLAND (Kamioka Liquid Scintillator Antineutrino Detector)} & \textsc{2009 - 2016}\\
	%Research Assistant, University of Hawaii & \textsc{Aug. 2009 - Apr. 2016}\\
	\multicolumn{2}{@{}l@{}}{\begin{minipage}[t]{\linewidth}
		\textit{Research Assistant, University of Hawaii at Manoa}
		\begin{itemize}
			\item Led unprecedented topological track reconstruction and
				particle ID studies in scintillator and applied these
				techniques using never before observed T2K events spilling into
				KamLAND
			\item Spearheaded development of novel directional neutrino
				detection technique in scintillator and demonstrated with data
				for the first time that this can be used for indirect dark
				matter search in scintillator
			\item Was responsible for high energy
				($\gtrsim$\SI{1}{\giga\electronvolt}) energy calibration using
				cosmic ray muons and applying this to neutrino analysis for
				the first time
		\end{itemize}
	\end{minipage}}\\
	\textsc{Mini-TimeCube (Portable Neutrino Detector)} & \textsc{2009 - 2016}\\
	\multicolumn{2}{@{}l@{}}{\begin{minipage}[t]{\linewidth}
		\textit{Research Assistant, University of Hawaii at Manoa}
		\begin{itemize}
			\item Led development of Geant4 detector simulation and guided
				detector design during construction
			%\item Led development of Geant4 detector simulation and
			%	mentored 3 undergraduate students to contribute to the overall
			%	detector design
			%\item Led critical role in directional neutrino/neutron
			%	reconstruction algorithm
			\item Was responsible for background studies associated with long
				lived cosmogenic isotopes \ce{8}He/\ce{9}Li, to quantitatively
				determine effect on detector live time
			\item Spearheaded trade studies for \ce{Li}/\ce{B} neutron capture
				dopants in scintillator for directional neutrino detection
				algorithm
		\end{itemize}
	\end{minipage}}\\
	%\textbf{CUORE (Cryogenic Underground Observatory for Rare Events)} & \textsc{Apr. 2016 -} \textit{Current}\\
	\textsc{CUORE (Cryogenic Underground Observatory for Rare Events)} & \textsc{Apr. 2016 -} \textit{Current}\\
	\multicolumn{2}{@{}l@{}}{\begin{minipage}[t]{\linewidth}
		\textit{Post-doctoral Scholar, University of California, Los Angeles (UCLA)}
		\begin{itemize}
			%\item Spearheading development of precision alpha background
			%	modeling in collaboration with a graduate student with goal for
			%	further background reduction to cover inverted neutrino mass
			%	hierarchy for $0\nu\beta\beta$ decay
			\item Spearheading development of precision alpha background
				modeling with goal for further background reduction to cover
				inverted neutrino mass hierarchy for $0\nu\beta\beta$ decay
			%\item Mentored and worked with 2 undergraduate students for
			%	investigation of shielding structures to mitigate $\gamma$ and
			%	beta backgrounds for next generation $0\nu\beta\beta$ decay
			%	searches requiring ultra-low background levels
			\item Implemented bolometer thermal model to create mock data for
				first data taking of CUORE
		\end{itemize}
	\end{minipage}}\\
\end{tabularx}

%----------------------------------------------------------------------------------------
%	LEADERSHIP AND WORK EXPERIENCE
%----------------------------------------------------------------------------------------

\section{Leadership and Teamwork}

\noindent\begin{tabularx}{\linewidth}{@{}lR@{}}
%\textsc{Aug. 2007 - May. 2009} & Teaching Assistant, University of Hawaii at Manoa\\
	\textsc{Mentor}, \textit{UCLA} & \textsc{2016 -} \textit{Current}\\
	\multicolumn{2}{@{}l@{}}{\begin{minipage}[t]{\linewidth}
		\begin{itemize}
			\item Taught weekly Geant4 simulation tutorials to \num{3}
				PhD students and \num{3} undergraduate students for 1 semester,
				students are now able to take on simulation tasks and
				collaborate in the group
			\item Led weekly Physics paper discussion groups for \num{3} PhD
				students, and promoted team work to increase dialogue and
				productivity within team
		\end{itemize}
	\end{minipage}}\\
	\textsc{Teaching Assistant}, \textit{University of Hawaii at Manoa} & \textsc{2007 - 2009}\\
	\multicolumn{2}{@{}l@{}}{\begin{minipage}[t]{\linewidth}
		\begin{itemize}
			\item Planned classwork and taught 2 weekly undergraduate Physics
				Laboratory classes of over \num{20} students each for 3
				semesters, received ``excellent'' reviews
			\item Mentored undergraduate students in undergraduate Physics
				classwork for 2 hours each week for 3 semesters, got students
				repeatedly seeking my particular tutoring
		\end{itemize}
	\end{minipage}}\\

%------------------------------------------------

%%\textsc{Jan. 2003 - Mar. 2006} & Interpreter and Teacher\\
%	\textbf{Interpreter and Teacher} & \textsc{2003 - 2006}\\
%	\multicolumn{2}{X}{Part time English lecturer for Korean undergraduate students.}\\
%	\multicolumn{2}{X}{Part time contributing reporter and translator for campus magazine.}\\
%	\multicolumn{2}{X}{Spontaneous trilingual interpreter for W-CARP International Education Conference.}\\
%	\multicolumn{2}{X}{Part time translator for magazine Today's World.}\\
\end{tabularx}

%----------------------------------------------------------------------------------------
%	SKILLS 
%----------------------------------------------------------------------------------------

\section{Skills}

\noindent\begin{tabularx}{\linewidth}{@{}rl}
	Programming Languages: & Proficient in C++, C, Python, Fortran, Mathematica, Bash\\
	Software/Tools: & \textsc{GENIE} neutrino event generator, \textsc{DarkSUSY}, \textsc{ROOT}, \textsc{Geant4}\\
	%Software/Tools: & \textsc{ROOT}, \textsc{Geant4}, \textsc{Pads}, \textsc{AutoCAD}\\
	Human Languages: & English (native), Japanese/Korean (trilingual proficiency)\\
\end{tabularx}

%----------------------------------------------------------------------------------------
%	LANGUAGES
%----------------------------------------------------------------------------------------

%\section{Languages}
%
%\noindent\begin{tabular}{rl}
%	\textsc{English}, \textsc{Japanese}, \textsc{Korean}\\
%\end{tabular}
%
%\newpage

%----------------------------------------------------------------------------------------
%	POSTERS AND TALKS
%----------------------------------------------------------------------------------------

\clearpage
\section{Invited Talks and Presentations}

\noindent\begin{tabularx}{\linewidth}{@{{}\textbullet\enskip}X@{\quad}r@{}}
	Division of Nuclear Physics, Pittsburgh/Carnegie Mellon University \newline Talk: \textsc{CUORE and Background Reduction Case Studies for CUPID} & Oct 2017 \\
	Conference on Science at SURF, South Dakota \newline Invited talk: \textsc{Status of the CUORE $0\nu\beta\beta$ Decay Search} & May 2017 \\
	Fermilab - Frontiers of Liquid Scintillator Technology \newline Invited talk: \textsc{Particle ID and event reconstruction algorithms in scintillator} & Mar 2016 \\
	DOE project review, Honolulu, Hawaii \newline Talk: \textsc{High Energy Analysis and Application to Dark Matter Search at KamLAND} & Jul 2015 \\
	Neutrino, Kyoto, Japan \newline Poster: \textsc{Indirect Dark-Matter Detection Through KamLAND} & Jun 2012 \\
	University of Hawaii Campus Open-house \newline Talks: \textsc{What is a Neutrino?}, \textsc{mini-TimeCube: The World's Smallest Neutrino Detector} & Nov 2010, 2011 \\
	Applied Antineutrino Physics, Sendai, Japan \newline Talk: \textsc{mini-TimeCube: A Portable Directional Neutrino Detector} & Aug 2010 \\
	DOE project review, Honolulu, Hawaii \newline Talk: \textsc{KamLAND Summary} & Sep 2009 \\
	Fermilab - International Neutrino Summer School \newline Talk: \textsc{Student presentation: How to solve $\theta_{23}$ degeneracy} & Jul 2009 \\
\end{tabularx}

%----------------------------------------------------------------------------------------
%	PUBLICATIONS
%----------------------------------------------------------------------------------------

\clearpage
\renewcommand\refname{Publications} %changes default name to Publications

\addbibresource{publications.bib}
\nocite{*} %lists everything in the .bib file
%\bibliographystyle{plain} %hundreds of styles to choose from
%\bibliography{publications.bib} %name of your .bib file
%\printbibliography[title=Alphabetic]
%\bibliographystyle{habbrvyr}
%\bibliography{publications.bib} %name of your .bib file
\printbibliography


%\section{Publications}
%
%\noindent\begin{tabular}{rp{11cm}}
%	\multicolumn{2}{l}{\textsc{mini-TimeCube}} \\
%	2015 (expected) & V.A. Li et al., \textsc{mini-TimeCube}, RSI Invited Review\\
%	\multicolumn{2}{c}{} \\
%	\multicolumn{2}{l}{\textsc{KamLAND}} \\
%	2015 (expected) & K. Asakura et al., \textsc{Search for the Proton Decay
%Mode $p \rightarrow \overline{\nu} K^{+}$ with KamLAND}, Phys. Rev. D\\
%	Mar. 2015 & K. Asakura et al., \textsc{Study of electron anti-neutrinos
%	associated with gamma-ray bursts using KamLAND}, arXiv:1503.02137v1\\
%	Feb. 2015 & T.I. Banks et al., \textsc{A compact ultra-clean system for
%	deploying radioactive sources inside the KamLAND detector},
%	10.1016/j.nima.2014.09.068\\
%	Jan. 2015 & C. Lane et al., \textsc{A new type of Neutrino Detector for
%	Sterile Neutrino Search at Nuclear Reactors and Nuclear Nonproliferation
%	Applications}, arXiv:1501.06935v1\\
%	May 2014 & A. Gando et al., \textsc{7Be Solar Neutrino Measurement with
%	KamLAND}, arXiv:1405.6190v1\\
%	Aug. 2011 & S. Abe et al., \textsc{Measurement of the 8B Solar Neutrino
%	Flux with the KamLAND Liquid Scintillator Detector},
%	10.1103/PhysRevC.84.035804\\
%	Aug. 2011 & J. Kumar, J.G. Learned, M. Sakai, S. Smith,
%	\textsc{Dark Matter Detection With Electron Neutrinos in Liquid
%	Scintillation Detectors}, Phys. Rev. D84 (2011) 036007\\
%\end{tabular}

%%----------------------------------------------------------------------------------------
%%   STATEMENT OF TEACHING PHILOSOPHY AND EXPERIENCE
%%----------------------------------------------------------------------------------------
%
%\clearpage
%\section{Statement of Teaching Philosophy and Experience}
%
%Throughout my academic career, I have been heavily involved in teaching and
%mentoring students from a wide range of cultures and backgrounds. This has
%included working as a teaching assistant during my graduate studies at
%University of Hawaii at Manoa where I organized and led 2 undergraduate Physics
%Laboratory classes of over 20 students each for a total of 3 semesters. In
%addition, I have also led weekly, 2 hour long, one-on-one tutoring sessions for
%3 semesters. The ethnic and cultural atmosphere in Hawaii is one of the most
%diverse on the planet. The University of Hawaii system not only attracts
%students internationally from Asia and from around the globe, but also serves
%as the hub of higher learning for students coming from countries in the Pacific
%Rim such as the Marshall Islands and Micronesia. In this cultural melting pot,
%instructors are required to effectively convey information in an inclusive and
%efficient way while understanding each of the students' cultures and needs as
%best as possible. For example, these needs may include supplemental instruction
%if English is not the student's primary language, or additional attention if
%the student comes from a background where little emphasis is placed on STEM
%fields. Student evaluations for the effectiveness of my teaching and ability to
%communicate were ``excellent''.
%
%My philosophy for teaching undergraduate Physics is to never get bogged down by
%the theory or equations. The primary driving force of a person's ability to
%learn is one's own self motivated curiosity. No other incentive to learn is
%more powerful and longer lasting than this. Therefore an instructor's priority
%at the undergraduate level should be focused on encouraging and cultivating the
%student's interest and curiosity. When I begin a class or introduce a new
%concept to students, I never begin with equations. I always first show them the
%experiment itself or introduce the context in an illustrative way. This is to
%engage the students' interest and entice their curiosity before going into
%anything that may slow them down and deter their motivation. Curiosity is where
%science is born from and curiosity is what drives science forward. This is true
%whether one is an undergraduate student or a Nobel laureate.
%
%In society today, there is an unfortunate misconception that Physics is a
%difficult subject and that it is only for \textit{smart} students. Nothing
%could be further from the truth. When students feel that they are not good
%enough or smart enough to follow the class, it is important to let them know
%that Physics is not about intelligence, but about curiosity driven diligence
%and hard work. All professional Physicists today stand on the shoulders of the
%giants of the past, and almost no research is done solely alone, but through
%collaboration and teamwork. No one is \textit{smart} enough to do all the work
%by one's self. I usually use myself as an example. I could not attend a leading
%world-class research institute during my undergraduate school days, but I
%believe through diligence and hard work, I have had the opportunity to work
%with some of the most talented and innovative Physicists in the field.
%
%Finally, my experience in mentoring students has also included teaching Geant4
%Monte Carlo particle simulations to 3 undergraduate and 3 Ph.D. students at the
%University of California, Los Angeles (UCLA). My method for teaching this
%simulation coding toolkit is to learn together with the students by going into
%the gritty details of the code together. This is because learning how to code
%is not just about acquiring the skill to code, but about learning the culture
%of coding. By this I mean that there are certain useful tools and methods of
%accomplishing a task that can only be learned through going into the gritty
%details and getting \textit{dirty} with someone more knowledgeable than you
%instead of learning from a straightforward top-down approach. This can be
%summed up in the popular saying: ``looking at someone else coding is the best
%way to learn how to code''. Through this method my students have grown to
%successfully been able to take on simulation tasks of their own and contribute
%effectively to the collaboration.

%----------------------------------------------------------------------------------------
%	STATEMENT OF RESEARCH
%----------------------------------------------------------------------------------------

\clearpage
\section{Statement of Research}

I developed a novel directional event reconstruction algorithm for high-energy
$\gtrsim$\si{\giga\electronvolt} scale neutrinos while working with KamLAND
(Kamioka Liquid Scintillation Antineutrino Detector), and demonstrated with
data that this technique can be applied to indirect dark matter search by
looking for a directional flux of neutrinos from the core of the Sun and Earth.
Studies done with Monte Carlo suggest that the accuracy of deducing the
neutrino direction using this new method is better than that of water-Cherenkov
detectors (the conventional method for directional neutrino detection) by
$\sim$\SI{10}{\degree} in this energy regime. This method was verified using
never before observed neutrino events spilling into KamLAND from the T2K
neutrino beam-line. The results were consistent with expectation. According to
my knowledge, this is the first ever physics application of neutrino
directionality in scintillator.

My work with KamLAND further involved demonstration of topological event
imaging techniques, originally developed in the LENA (Low Energy Neutrino
Astronomy) collaboration, using data for the first time. The
$\sim$\SI{3.5}{\nano\second} timing resolution of the PMTs (photomultiplier
tubes) employed in KamLAND are not good enough to do a detailed imaging.
Nevertheless $\gtrsim$\si{\giga\electronvolt} muon tracks can be well resolved
as well as the overall direction of the final state particles to resolve the
incoming neutrino direction. In addition $\frac{\mathrm{d}E}{\mathrm{d}x}$
profiles were investigated to perform unprecedented particle ID studies in
scintillator.

The above studies required me to use the GENIE neutrino event generator and
perform unusually high energy (\SIrange{1}{100}{\giga\electronvolt}) Geant4
Monte Carlo simulations in scintillator. The difficulty that lies here is in
the large amounts of computing power necessary for a faithful reproduction of
the physics with such a vast number of photons produced. The way I mitigated
this problem was to do an initial \textit{sparse} simulation only using certain
physics in crucial detector parts first, and then, to control the evolution of
the random number generator such that individual events can be hand picked at a
later time to further simulate additional physics or detector volumes. In this
way, I was able to reduce the computing time to a minimum while still
incorporating precision physics. A paper for my work is currently under
preparation.

In addition, I have worked as the lead Geant4 simulation designer for the
mini-TimeCube collaboration at University of Hawaii at Manoa. mini-TimeCube is
an ambitious project to build the world's smallest portable neutrino detector.
In this project, I mentored 3 undergraduate students and worked in
collaboration with them to conduct case studies for optimizing the detector
design, test candidate neutron capture doping elements in plastic scintillator,
and simulate the response of the multi-channel-plate (MCP) PMTs deployed in the
detector. These studies were used during construction of the detector, and to
develop directional algorithms that are now being tested in analyses of
neutrons from test sources as well as neutrinos from nuclear reactors.
%I have also conducted simulation studies for cosmic-ray muons and long-lived
%cosmogenic background isotopes such as \ce{^{8}He} and \ce{^{9}Li}. These
%backgrounds are extremely difficult to tag due to their long life-time ($>
%\sim\si{\second}$ scale) and travel distances. The studies have been vital to
%the project. 
Working with the mini-TimeCube project has further involved designing and
fabricating PCB boards as well as contributing to the FPGA firmware for the
readout electronics. A paper summarizing our accomplishments was published in
2016 (V. A. Li et al. Invited Article: miniTimeCube. Rev. Sci. Instrum.,
87(2):021301, 2016, 1602.01405).

I have been involved with the CUORE (Cryogenic Underground Observatory for Rare
Events) experiment at the University of California, Los Angeles since 2016. The
main objective of the CUORE experiment is to hunt for neutrinoless double beta
($0\nu\beta\beta$) decay using \ce{^{130}Te} using a tonne-scale array of
bolometers in a cryogenic environment. My current role is to lead the
development of a precision alpha background model. The energy spectrum of the
backgrounds in the so-called alpha region ($\gtrsim
\SI{2.5}{\mega\electronvolt}$) exhibits peculiar features that, if understood
correctly will better explain the types of background sources and their
distributions in the detector parts. This can help us to better understand our
backgrounds and to extrapolate this knowledge to the energy region of interest
(\SIrange{2465}{2575}{\kilo\electronvolt}) for $0\nu\beta\beta$ decay search in
\ce{^{130}Te}. I have also been heavily involved in the thermal modeling and
optimization of the signal processing for the recent upgrade from CUORE-0 to
CUORE which increased the detector mass by a factor of almost 20. A paper for
our first $0\nu\beta\beta$ analysis using CUORE data was submitted for
publication to PRL in late 2017, and is currently under review
(https://arxiv.org/abs/1710.07988)



%----------------------------------------------------------------------------------------
%	REFERENCES
%----------------------------------------------------------------------------------------

\clearpage
\section{References}

\noindent\begin{tabular}{rp{12cm}}
\multicolumn{2}{l}{\footnotesize{Supplied upon request or please contact in
	person.}}\\
\multicolumn{2}{c}{}\\
Huan Z. \textsc{Huang} & Professor, University of California, Los Angeles, +1-310-825-9297\\
& \href{mailto:huang@physics.ucla.edu}{\nolinkurl{huang@physics.ucla.edu}}\\
John G. \textsc{Learned} & Professor, University of Hawaii at Manoa, +1-808-956-2964\\
& \href{mailto:jgl@phys.hawaii.edu}{\nolinkurl{jgl@phys.hawaii.edu}}\\
Yury \textsc{Kolomensky} & Professor, University of California, Berkeley, +1-510-642-9619\\
& \href{mailto:ygkolomensky@lbl.gov}{\nolinkurl{ygkolomensky@lbl.gov}}\\
Brian K. \textsc{Fujikawa} & Staff Scientist, Lawrence Berkeley National Laboratory, +1-510-486-4398\\
& \href{mailto:bkfujikawa@lbl.gov}{\nolinkurl{bkfujikawa@lbl.gov}}\\
Lindley \textsc{Winslow} & Jerrold R. Zacharias Assistant Professor, MIT, +1-617-253-2332\\
& \href{mailto:lwinslow@mit.edu}{\nolinkurl{lwinslow@mit.edu}}\\ 
Thomas \textsc{O'Donnell} & Assistant Professor, Virginia Tech, +1-540-231-3308\\
& \href{mailto:todonnell@vt.edu}{\nolinkurl{todonnell@vt.edu}}\\

%Kunio \textsc{Inoue} & Professor, Tohoku Univ./RCNS, +81-22-795-6727,
%\href{mailto:inoue@awa.tohoku.ac.jp}{\nolinkurl{inoue@awa.tohoku.ac.jp}}\\
%
%Jason \textsc{Kumar} & Assoc. Professor, Univ. of Hawaii at Manoa, +1-808-956-2972,
%\href{mailto:jkumar@phys.hawaii.edu}{\nolinkurl{jkumar@phys.hawaii.edu}}\\
%
%Jelena \textsc{Maricic} & Assoc. Professor, Univ. of Hawaii at Manoa, +1-808-956-7176,
%\href{mailto:jelena@phys.hawaii.edu}{\nolinkurl{jelena@phys.hawaii.edu}}\\

%Adam \textsc{Bernstein} & P.I. Applied Antineutrino Physics, LLNL,
%\href{mailto:bernstein3@llnl.gov}{\nolinkurl{bernstein3@llnl.gov}}\\
%\multicolumn{2}{l}{\footnotesize{(note: I have met Dr. Bernstein once during AAP
%2010 Sendai, so perhaps he knows me least within the listed referrers.)}}\\
\end{tabular}

%----------------------------------------------------------------------------------------

\end{document}
