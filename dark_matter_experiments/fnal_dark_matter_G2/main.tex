%% start of file `template.tex'.
%% Copyright 2006-2013 Xavier Danaux (xdanaux@gmail.com).
%
% This work may be distributed and/or modified under the
% conditions of the LaTeX Project Public License version 1.3c,
% available at http://www.latex-project.org/lppl/.


\documentclass[11pt,a4paper,sans]{moderncv}        % possible options include font size ('10pt', '11pt' and '12pt'), paper size ('a4paper', 'letterpaper', 'a5paper', 'legalpaper', 'executivepaper' and 'landscape') and font family ('sans' and 'roman')

% moderncv themes
\moderncvstyle{casual}                             % style options are 'casual' (default), 'classic', 'oldstyle' and 'banking'
\moderncvcolor{blue}                               % color options 'blue' (default), 'orange', 'green', 'red', 'purple', 'grey' and 'black'
%\renewcommand{\familydefault}{\sfdefault}         % to set the default font; use '\sfdefault' for the default sans serif font, '\rmdefault' for the default roman one, or any tex font name
%\nopagenumbers{}                                  % uncomment to suppress automatic page numbering for CVs longer than one page

% character encoding
\usepackage[utf8]{inputenc}                       % if you are not using xelatex ou lualatex, replace by the encoding you are using
%\usepackage{CJKutf8}                              % if you need to use CJK to typeset your resume in Chinese, Japanese or Korean

% adjust the page margins
\usepackage[scale=0.75]{geometry}
%\setlength{\hintscolumnwidth}{3cm}                % if you want to change the width of the column with the dates
%\setlength{\makecvtitlenamewidth}{10cm}           % for the 'classic' style, if you want to force the width allocated to your name and avoid line breaks. be careful though, the length is normally calculated to avoid any overlap with your personal info; use this at your own typographical risks...
\usepackage[version=4]{mhchem}

% personal data
\name{Michinari}{Sakai}
\title{Resumé title}                               % optional, remove / comment the line if not wanted
\address{475 Portola Plaza \#5-123B}{Los Angeles, CA 90095-1547}{USA}% optional, remove / comment the line if not wanted; the "postcode city" and and "country" arguments can be omitted or provided empty
\phone[mobile]{+1~(808)~206~4357}                   % optional, remove / comment the line if not wanted
%\phone[fixed]{+2~(345)~678~901}                    % optional, remove / comment the line if not wanted
%\phone[fax]{+3~(456)~789~012}                      % optional, remove / comment the line if not wanted
\email{michsakai@ucla.edu}                               % optional, remove / comment the line if not wanted
%\homepage{www.johndoe.com}                         % optional, remove / comment the line if not wanted
%\extrainfo{additional information}                 % optional, remove / comment the line if not wanted
%\photo[64pt][0.4pt]{picture}                       % optional, remove / comment the line if not wanted; '64pt' is the height the picture must be resized to, 0.4pt is the thickness of the frame around it (put it to 0pt for no frame) and 'picture' is the name of the picture file
%\quote{Some quote}                                 % optional, remove / comment the line if not wanted

% to show numerical labels in the bibliography (default is to show no labels); only useful if you make citations in your resume
%\makeatletter
%\renewcommand*{\bibliographyitemlabel}{\@biblabel{\arabic{enumiv}}}
%\makeatother
%\renewcommand*{\bibliographyitemlabel}{[\arabic{enumiv}]}% CONSIDER REPLACING THE ABOVE BY THIS

% bibliography with mutiple entries
%\usepackage{multibib}
%\newcites{book,misc}{{Books},{Others}}
%----------------------------------------------------------------------------------
%            content
%----------------------------------------------------------------------------------
\begin{document}
%-----       letter       ---------------------------------------------------------
% recipient data
\recipient{Dr. Dan Bauer}{Fermilab MS 127\\P.O. Box 500\\Batavia, IL 60510}
\date{November 05, 2017}
\opening{Dear Dr. Dan Bauer,}
\closing{Best wishes,}
\enclosure[Attached]{curriculum vit\ae{}, research statement, publication list, list of referees}          % use an optional argument to use a string other than "Enclosure", or redefine \enclname
\makelettertitle

I recently received word of an opening at the Fermilab Center for Particle
Astrophysics (FCPA) from one of my colleagues, and I would like to apply for
the Postdoctoral Research Associate position working on the next generation
(G2) of dark matter direct detection experiments.

I worked at the University of California, Los Angeles as a postdoctoral
researcher from April, 2016, with CUORE (Cryogenic Underground Observatory for
Rare Events)  at LNGS (Laboratori Nazionali del Gran Sasso), Italy working on
Monte Carlo studies of backgrounds coming from radioactive contamination of the
detector and modeling thermal pulse signals of bolometers in a cryogenic
environment, as well as various signal processing algorithms.

I received my Ph.D. from the University of Hawaii in 2016 working on KamLAND
(Kamioka Liquid Scintillator Antineutrino Detector) in Kamioka, Japan. My
dissertation was on indirect dark matter search by looking for a directional
excess of neutrino fluxes from the core of the Sun and Earth. According to my
understanding, this work is the first ever physics application of neutrino
directionality in scintillator. A paper for this result is currently under
preparation.

My interest and experience in ultra-low background experiments in a cryogenic
environment, as well as my previous work on dark matter places me in a unique
position to apply for your position, and I believe that I can make a
significant contribution to your group. Please feel free to contact me if you
have any questions or need further information.

Thank you for your consideration.

\makeletterclosing

\end{document}


%% end of file `template.tex'.
